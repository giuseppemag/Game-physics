\documentclass{beamer}
\usetheme{Warsaw}
\usepackage{nhtvslides}
\usepackage{graphicx}
\usepackage{listings}
\lstset{language=CAML,
basicstyle=\ttfamily\footnotesize,
frame=shadowbox,
breaklines=true}
\usepackage[utf8]{inputenc}

\title{Integration, rotation matrices, and quaternions}

\author{Dr. Giuseppe Maggiore}

\institute{NHTV University of Applied Sciences \\ 
Breda, Netherlands}

\date{}

\begin{document}
\maketitle

\begin{frame}{Table of contents}
\tableofcontents
\end{frame}

\section{Integration and rotation}
\begin{slide}{Integration and rotation}{Integration and rotation}{
\item We integrate $L$ from $\tau$
\item We integrate $w$ from $L$ and $J^-1$
\item We integrate $R$ from $w$
}\end{slide}

\begin{slide}{Integration and rotation}{Issues}{
\item \textit{Issue 1:} recomputing $J$ from every particle of the body and at every tick is too slow
\item \textit{Issue 2:} integrating $R$ slowly ``breaks it'', meaning that $[T\ N\ B]$ is not a rotation anymore
}\end{slide}

\section{Caching J}
\begin{slide}{Caching J}{Issue 1: body shape}{
\item $J$ represents the tendency of the body to resist rotation
\item But the body just moves and rotates, it does not change shape
\item We should be able to simply recompute $J$ from its initial value and the rotation
}\end{slide}

\begin{frame}{Caching J}
\begin{block}{$J$ from $J_{body}$}
\begin{eqnarray}
J &=& \sum_i(|r_i(t)|^2I - r_i(t) r_i^T(t)) \\
&=& \sum_i(|R r_i(t_0)|^2I - R r_i(t_0) r_i^T(t_0) R^T) \\
&=& \sum_i(|r_i(t_0)|^2I - R r_i(t_0) r_i^T(t_0) R^T) \\ % rotation keeps length
&=& \sum_i(|r_i(t_0)|^2 R R^T - R r_i(t_0) r_i^T(t_0) R^T) \\ % R R^T = I
&=& \sum_i(R |r_i(t_0)|^2 R^T - R r_i(t_0) r_i^T(t_0) R^T) % scalar and matrix multiplication is commutative
\end{eqnarray}
\end{block}
\end{frame}

\begin{frame}{Caching J}
\begin{block}{$J$ from $J_{body}$}
\begin{eqnarray}
J &=& \sum_i(R |r_i(t_0)|^2 R^T - R r_i(t_0) r_i^T(t_0) R^T) \\ % scalar and matrix multiplication is commutative
&=& \sum_i R(|r_i(t_0)|^2 - r_i(t_0) r_i^T(t_0)) R^T \\ % grouping
&=& R \left( \sum_i (|r_i(t_0)|^2 - r_i(t_0) r_i^T(t_0)) \right) R^T \\ % grouping
&=& R J_{body} R^T
\end{eqnarray}
\end{block}
\end{frame}

\begin{slide}{Caching J}{$J^{-1}$ from $J_{body}{-1}$}{
\item We need $J^{-1}$ more than $J$
\item We can compute it without an inversion though
\item We compute (just once) $J_{body}^{-1}$
\item $J^{-1} (R J_{body} R^T)^{-1} = R J_{body}^{-1} R^T$
}\end{slide}

\section{Orthonormalization of $R$}
\begin{slide}{Orthonormalization of $R$}{Issue 2: $R$ is a rotation matrix}{
\item $R = [T\ N\ B]$
\item The vectors $T$, $N$, and $B$ must be of unit length and orthonormal ($T \cdot N = 0, T \cdot B = 0, \dots$)
\item As we integrate $R$, this stops being true
\item $R$ stops being just a rotation matrix, and also incorporates some scale and some skew
\item This is very bad!
}\end{slide}

\begin{frame}{Boxes stop being boxes}
\center
\includegraphics[height=5cm]{Pics/Kill_me_please.png}
\end{frame}

\begin{slide}{Orthonormalization of $R$}{Gram-Schimdt method}{
\item $\hat R = [\hat T\ \hat N\ \hat B]$
\item We normalize $T = \frac{\hat T}{|\hat T|}$
\item We remove the projection of $\hat N$ over $T$: $\hat N' = \hat N - \hat N (\hat N \cdot T)$
\item We normalize $N = \frac{\hat N'}{|\hat N'|}$
\item We compute $B = T \times N$
}\end{slide}

\begin{slide}{Orthonormalization of $R$}{Gram-Schimdt method}{
\item Lot of useless computation
\item Matrix representation very redundant
\item $\hat B$ is completely unneeded and is ignored!
}\end{slide}

\section{Using quaternions}
\begin{slide}{Using quaternions}{About quaternions}{
\item We just need a single rotation around an axis (which is an arbitrary rotation in 3D)
\item Less degrees of freedom means less drift and less need for normalization
}\end{slide}

\begin{slide}{Using quaternions}{Quaternion primer}{
\item Let us consider a rotation matrix around the Z axis
\item $R_0 = \left[ \begin{matrix}
\cos \theta & -\sin \theta & 0 \\
\sin \theta & \cos \theta & 0 \\
0 & 0 & 1\\
\end{matrix} \right] = 
\left[ \begin{matrix}
c & -s & 0 \\
s & c & 0 \\
0 & 0 & 1 \\
\end{matrix} \right]$
}\end{slide}

\begin{slide}{Using quaternions}{Quaternion primer}{
\item Any vector $v$ on the XY plane rotated by $R_0$ remains on XY, rotated around Z by an angle $\theta$
\item Any vector $v$ on the Z axis rotated by $R_0$ remains (identical) on Z
}\end{slide}

\begin{slide}{Using quaternions}{Quaternion primer}{
\item Let us transform this rotation into an arbitrary rotation $R_1$ around an axis $d$ and of angle $\theta$
\item We consider an \textit{orthonormal basis} $a,b,d$
\item We perform a \textit{basis change} of $R_0$ onto $a,b,d$
}\end{slide}

\begin{slide}{Using quaternions}{Quaternion primer}{
\item We need to build $R_1$ such that
\begin{itemize}
\item $R_1 a = ca + sb$
\item $R_1 b = -sa + cb$
\item $R_1 d = d$
\end{itemize}
\item That is $R_1 \underbrace{[a\ b\ d]}_P = \underbrace{[a\ b\ d]}_P R_0$
\item $P^{-1} = P^T$, so $R_1 = P R_0 P^T = c(aa^T + bb^T) + s(ba^T - ab^T) + dd^T$
}\end{slide}

\begin{slide}{Using quaternions}{Quaternion primer}{
\item $R_1 = c(aa^T + bb^T) + s(ba^T - ab^T) + dd^T$
\item This means that we can now construct a rotation matrix from an axis and an angle
\item Unfortunately we also need two ``dummy'' vectors $a$ and $b$
\item Any pair of those suffices, as long as they are correctly related to $d$
\item We now remove the explicit dependency on them
}\end{slide}

\begin{slide}{Using quaternions}{Quaternion primer}{
\item Consider a vector expressed in relationship to $a,b,d$: $v = \underbrace{\alpha}_{a \cdot v} a + \underbrace{\beta}_{b \cdot v} b + \underbrace{\delta}_{d \cdot v} d$
\item We now consider two arbitrary but useful quantities:
\item $d \times v = d \times (\alpha a + \beta b + \delta d) = \alpha d \times a + \beta d \times b + \delta d \times d = -\beta a + \alpha b = Dv = \left[ \begin{matrix}
0 & -d_z & d_y \\
d_z & 0 & -d_x \\
-d_y & d_x & 0 \\
\end{matrix} \right] v$
\item $d \times (d \times v) = d \times (-\beta a + \alpha b) = -\alpha a - \beta b = D^2v$
}\end{slide}

\begin{slide}{Using quaternions}{Quaternion primer}{
\item Now let us study the progression of $Iv, Dv, D^v$
\item $Iv = v = \alpha a + \beta b + \delta d = aa^Tv + bb^Tv + dd^Tv$, so $I = aa^T + bb^T + dd^T$
\item $Dv = \alpha b - \beta a = ba^Tv - ab^Tv$, so $D = ba^T-ab^T$
\item $D^2v = -\alpha a - \beta b = \delta d - v = dd^Tv - v$, so $D^2 = dd^T - I$
}\end{slide}

\begin{slide}{Using quaternions}{Quaternion primer}{
\item We combine those three into $R_1 = c(aa^T + bb^T) + s(ba^T - ab^T) + dd^T$
\item $I = aa^T + bb^T + dd^T$, $D = ba^T-ab^T$, $D^2 = dd^T - I$
\pause 
\item $R_1 = c(I - dd^T) + sD + dd^T$
\item $R_1 = I + sD + (1-c)D^2$
}\end{slide}

\begin{slide}{Using quaternions}{Quaternion primer}{
\item We can now compute $R_1 = I + sD + (1-c)D^2$ from just $d, \theta$
\item Moreover, we can rotate a vector without even building $R_1$
\begin{eqnarray}
R_1v &=& (I + s D + (1-c)D^2)v \\
&=& Iv + s D v + (1-c)D^2 v \\
&=& v + s d\times v + (1-c)d \times (d \times v)
\end{eqnarray}
}\end{slide}

\begin{slide}{Using quaternions}{Quaternion primer}{
\item Any representation of $d$ and $\theta$ would be fine
\item A particularly convenient one is $[\cos \frac{\theta}{2}, \sin \frac{\theta}{2} d]$
}\end{slide}

\section{Time derivative of a quaternion}
\begin{slide}{Deriving quaternions}{Derivative of rotation matrix}{
\item We studied how to compute $\dot R$ from $R$ and $\omega$
\item If we represent rotations with quaternions, we need to compute $\dot q$
}\end{slide}

\begin{slide}{Deriving quaternions}{Derivative of quaternion}{
\item Given the current orientation $q(t_0)$ and angular velocity $\omega(t_0)$
\item Assuming very small $t-t_0$
\item The derivative of the original quaternion is $\frac{d}{dt}\underbrace{[\cos \frac{|\omega(t_0)|(t-t_0)}{2}, \frac{\omega(t_0)}{|\omega(t_0)|} \sin \frac{|\omega(t_0)|(t-t_0)}{2}]}_{\omega \text{ converted to a quaternion}} q(t_0)$
}\end{slide}

\begin{slide}{Deriving quaternions}{Derivative of quaternion}{
\item $\frac{d}{dt}[\cos \frac{|w(t_0)|(t-t_0)}{2}, \frac{w(t_0)}{|w(t_0)|} \sin \frac{|w(t_0)|(t-t_0)}{2}] q(t_0)$
\item We now derive the $w$ part ($q(t_0)$ is a constant) and replace $t$ with $t_0$ because we want to know the value of the derivative at the current time $t_0$
\item $[-\frac{|\omega(t_0)|}{2} \sin \frac{|\omega(t_0)|(t_0-t_0)}{2}, \frac{\omega(t_0)}{|\omega(t_0)|} \frac{|\omega(t_0)|}{2} \cos \frac{|\omega(t_0)|(t-t_0)}{2}] q(t_0)$
\item $[-\frac{|\omega(t_0)|}{2} \sin 0, \frac{\omega(t_0)}{|\omega(t_0)|} \frac{|\omega(t_0)|}{2} \cos 0] q(t_0)$
\item $[0, \frac{\omega(t_0)}{2}] q(t_0) = \frac{1}{2} [0, \omega(t_0)] q(t_0)$
}\end{slide}

\begin{slide}{Deriving quaternions}{Derivative of quaternion}{
\item At least remember this:
\item $\dot q = \frac{1}{2} [0, \omega] q$
}\end{slide}

\section{Kinematics source code}
\begin{frame}[fragile]{Kinematics source code}
\begin{lstlisting}
// initialization 
RigidBody* body[n]; 
double t = <your choice of initial time>; 
double dt = <your choice of time step>; 
for (int i = 0; i < n; ++i) { 
  // Set the initial state of the rigid bodies. 
  body[i] = new RigidBody(...);

// Part of the physics tick. 
for (i = 0; i < n; ++i) { 
  body[i].Update(t, dt);
  t += dt;
}
\end{lstlisting}
\end{frame}

\begin{frame}[fragile]{Kinematics source code}
\begin{lstlisting}
struct RigidBody {
  RigidBody (double m, matrix inertia, Function force, Function torque);

  // force/torque function format 
  typedef vector ( *Function ) (
    double, // time of application
    point, // position
    quaternion, // orientation
    ... // whole state of body, one var at a time
    )
  );

  // Runge-Kutta fourth order differential equation solver 
  void Update (double t, double dt);
\end{lstlisting}
\end{frame}

\begin{frame}[fragile]{Kinematics source code}
\begin{lstlisting}
protected: 
  // convert (Q,P,L) to (R,V,W) 
  void Convert (quaternion Q, vector P, vector L, matrix& R, vector& V, vector& W) const;
  // constant quantities 
  double m_mass, m_invMass; 
  matrix m_inertia, m_invInertia;
  // state variables 
  vector m_X; // position 
  quaternion m_Q; // orientation 
  vector m_P; // linear momentum 
  vector m_L; // angular momentum
\end{lstlisting}
\end{frame}

\begin{frame}[fragile]{Kinematics source code}
\begin{lstlisting}
  // derived state variables 
  matrix m_R; // orientation matrix 
  vector m_V; // linear velocity vector 
  m_W; // angular velocity
  // force and torque functions 
  Function m_force; Function m_torque;
};
\end{lstlisting}
\end{frame}

\begin{frame}[fragile]{Kinematics source code}
\begin{lstlisting}
void RigidBody::Convert (quaternion Q, vector P, vector L, matrix& R, vector& V, vector& W) const { 
  Q.ToRotationMatrix(R); 
  V = m_invMass*P; 
  W = R*m_invInertia*Transpose(R)*L; 
}
\end{lstlisting}
\end{frame}

\begin{frame}[fragile]{Kinematics source code}
\begin{lstlisting}
void RigidBody::Update (double t, double dt) { 
  double halfdt = 0.5 * dt, sixthdt = dt / 6.0;
  double tphalfdt = t + halfdt, tpdt = t + dt;
\end{lstlisting}
\end{frame}

\begin{frame}[fragile]{Kinematics source code}
\begin{lstlisting}
  vector XN, PN, LN, VN, WN; 
  quaternion QN; 
  matrix RN;
  // A1 = G(t,S0), B1 = S0 + (dt / 2) * A1 
  vector A1DXDT = m_V; 
  quaternion A1DQDT = 0.5 * m_W * m_Q; 
  vector A1DPDT = m_force(t,m_X,m_Q,m_P,m_L,m_R,m_V,m_W); 
  vector A1DLDT = m_torque(t,m_X,m_Q,m_P,m_L,m_R,m_V,m_W); 
  XN = m_X + halfdt * A1DXDT; 
  QN = m_Q + halfdt * A1DQDT; 
  PN = m_P + halfdt * A1DPDT; 
  LN = m_L + halfdt * A1DLDT; 
  Convert(QN,PN,LN,RN,VN,WN);
\end{lstlisting}
\end{frame}

\begin{frame}[fragile]{Kinematics source code}
\begin{lstlisting}
  // A2 = G(t + dt / 2,B1), B2 = S0 + (dt / 2) * A2 
  vector A2DXDT = VN; 
  quaternion A2DQDT = 0.5 * WN * QN; 
  vector A2DPDT = m_force(tphalfdt,XN,QN,PN,LN,RN,VN,WN); 
  vector A2DLDT = m_torque(tphalfdt,XN,QN,PN,LN,RN,VN,WN); 
  XN = m_X + halfdt * A2DXDT; 
  QN = m_Q + halfdt * A2DQDT; 
  PN = m_P + halfdt * A2DPDT; 
  LN = m_L + halfdt * A2DLDT; 
  Convert(QN,PN,LN,RN,VN,WN);
\end{lstlisting}
\end{frame}

\begin{frame}[fragile]{Kinematics source code}
\begin{lstlisting}
  // A3 = G(t + dt / 2,B2), B3 = S0 + dt * A3 
  vector A3DXDT = VN; 
  quaternion A3DQDT = 0.5 * WN * QN;

  vector A3DPDT = m_force(tphalfdt,XN,QN,PN,LN,RN,VN,WN); 
  vector A3DLDT = m_torque(tphalfdt,XN,QN,PN,LN,RN,VN,WN); 
  XN = m_X + dt * A3DXDT; 
  QN = m_Q + dt * A3DQDT; 
  PN = m_P + dt * A3DPDT; 
  LN = m_L + dt * A3DLDT; 
  Convert(QN,PN,LN,RN,VN,WN);
\end{lstlisting}
\end{frame}

\begin{frame}[fragile]{Kinematics source code}
\begin{lstlisting}
  // A4 = G(t + dt,B3), 
  // S1 = S0 + (dt / 6) * (A1 + 2 * A2 + 2 * A3 + A4) 
  vector A4DXDT = VN; 
  quaternion A4DQDT = 0.5 * WN * QN; 
  vector A4DPDT = m_force(tpdt,XN,QN,PN,LN,RN,VN,WN); 
  vector A4DLDT = m_torque(tpdt,XN,QN,PN,LN,RN,VN,WN); 
  m_X = m_X + sixthdt*(A1DXDT + 2.0*(A2DXDT + A3DXDT) + A4DXDT); 
  m_Q = m_Q + sixthdt*(A1DQDT + 2.0*(A2DQDT + A3DQDT) + A4DQDT); 
  m_P = m_P + sixthdt*(A1DPDT + 2.0*(A2DPDT + A3DPDT) + A4DPDT); 
  m_L = m_L + sixthdt*(A1DLDT + 2.0*(A2DLDT + A3DLDT) + A4DLDT); 
  Convert(m_Q,m_P,m_L,m_R,m_V,m_W);
}
\end{lstlisting}
\end{frame}


\begin{frame}{That's it}
\center
\fontsize{18pt}{7.2}\selectfont
Thank you!
\end{frame}

\end{document}
